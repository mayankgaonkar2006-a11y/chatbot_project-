\documentclass{article}
\usepackage[margin=1in]{geometry}

\begin{document}

\section{Initial Commit: Chatbot Project Setup}
The initial commit includes the basic chatbot functionality. The chatbot uses a simple natural language processing (NLP) model to respond to user input.

\subsection{How it Works}
The chatbot uses a dictionary-based approach to match user input to predefined responses. The NLP model is trained on a dataset of intents and responses, allowing the chatbot to generate responses based on user input.

\subsubsection{Code Explanation}
The code uses a combination of Python and NLP libraries to process user input and generate responses. The main components of the code include:
\begin{itemize}
    \item Intent recognition: The chatbot uses a machine learning model to classify user input into different intents.
    \item Response generation: The chatbot generates responses based on the recognized intent.
\end{itemize}

\subsection{References}
\begin{itemize}
    \item \href{https://github.com/Chando0185/Multiverse_of_100-_data_science_project_series/tree/main/End%20to%20End%20Chatbot%20using%20Python}{github link for reference}
    \item \href{https://www.youtube.com/watch?v=fVDOlz4gHDI&list=PLWyN7K28ZraQi1_7ILgiKAiY_FmGeQWbI&index=5}{YouTube Video: Chatbot Development Tutorial}
\end{itemize}
#if you cannot enter through the given link in github go to the youtube link and scroll down to the description you will get the github link

\end{document}
